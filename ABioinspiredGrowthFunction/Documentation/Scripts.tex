% Load packages {{{
\documentclass{article}
\usepackage[utf8]{inputenc}
\usepackage[colorlinks=true]{hyperref}
\usepackage[cmex10]{amsmath}
\usepackage{mdwmath}
\usepackage{mdwtab}
\usepackage{amsfonts}
\usepackage{mathrsfs}
\usepackage{bm}
\usepackage{subfig}
\usepackage[capitalise]{cleveref}
\usepackage{multirow}
\usepackage{tikz}
\usetikzlibrary{shapes.geometric,arrows}
\usepackage{soul}
\usepackage{float}
\usepackage{setspace}
\usepackage{natbib}
\usepackage{booktabs}

\usepackage{minted}
\usemintedstyle{tango}
% }}}

% Title and authors {{{
%Title
\title{Scripts}
%Authors
\author{David Hernández-Aristizábal}
% }}}

\begin{document}

% Settings {{{
% New colours
\definecolor{BlueHTML}{RGB}{0, 102, 204}
% Hyperref colour setup
\hypersetup{linkcolor = BlueHTML}
% }}}

\maketitle
\tableofcontents

% Structure {{{
\section{Main python scripts}
% RunAnalysis.py {{{
\subsection{Run analysis}
% Execution {{{
\subsubsection{Execution}
\begin{minted}{bash}
    python3 RunAnalysis.py -i PARAMS.json
\end{minted}
% }}}
% Required packages {{{
\subsubsection{Required packages}
\begin{minted}{python}
    # Python libraries
    import os
    import sys
    import json
    import getopt
    import time
    # In-house modules
    sys.path.append('../PythonUtilities')
    import AsterStudyUtilities as asus
\end{minted}
% }}}
% General function {{{
\subsubsection{General function}
\begin{minted}{python}
    # Read parameters
    params = json.load("PARAMS.json")
    # Create Code_Aster study
    study = asus.AsterStudy()
    # Set up study parameters
    study.PARAMETER = PARAMETER
    study.comm = ".../BioInspiredGrowth.comm"
    study.dumm = ".../BioInspiredGrowth.dumm"
    # Create EXPORT file
    study.CreateStudy()
    # Create initial geometry
    os.system("python3 .../InitialGeometry.py -i PARAMS.json")
    # Run Code_Aster study
    study.RunStudy(outSalome = True)
    while notFinish:
        # Remesh geometry
        os.system("python3 .../Remesh.py -i PARAMS.json")
        # Run Code_Aster study
        study.RunStudy(outSalome = True)

\end{minted}
% }}}
% }}}

% InitialGeometry.py {{{
\subsection{Initial geometry}
% Execution {{{
\subsubsection{Execution}
\begin{minted}{bash}
    python3 InitialGeometry.py -i PARAMS.json
\end{minted}
% }}}
% Required packages {{{
\subsubsection{Required packages}
\begin{minted}{python}
    # Python libraries
    import os
    import sys
    import json
    import getopt
    import gmsh
    import math
    import numpy
    # In-house modules
    sys.path.append('../PythonUtilities')
    import ReadGmsh as rgmsh
\end{minted}
% }}}
% General function {{{
\subsubsection{General function}
\begin{minted}{python}
    # Read parameters
    params = json.load("PARAMS.json")
    # Make the deformable geometry
    gmsh.initialize()
    wp_face = geo.addPlaneSurface([wp_loop])
    # Approximation of contact length
    fine_x = closestPoint()
    # Definition of the field of element sizes
    mesh.field.setAsBackgroundMesh(bacFieldNum)
    # Mesh of the deformable geometry
    mesh.generate(2)
    gmsh.write("geomwp.unv")
    gmsh.finalize()
    # Make the non-deformable geometry
    gmsh.initialize()
    base_face = geo.addPlaneSurface([base_loop])
    # Definition of the field of element sizes
    bsFields = rgmsh.BoxFieldsIncreasingSize(...fine_x...)
    # Mesh the non-deformable geometry
    mesh.generate(2)
    gmsh.write("geombs.unv")
    gmsh.finalize()
\end{minted}
% }}}
% }}}

% Remesh.py {{{
\subsection{Remesh}
% Execution {{{
\subsubsection{Execution}
\begin{minted}{bash}
    python3 Remesh.py -i PARAMS.json
\end{minted}
% }}}
% Required packages {{{
\subsubsection{Required packages}
\begin{minted}{python}
    # Python libraries
    import os
    import sys
    import json
    import getopt
    import gmsh
    import numpy
    # In-house modules
    sys.path.append('../PythonUtilities')
    import ReadGmsh as rgmsh
\end{minted}
% }}}
% General function {{{
\subsubsection{General function}
\begin{minted}{python}
    # Read parameters
    params = json.load("PARAMS.json")
    # Reconstruction of the deformable geometry
    gmsh.initialize()
    model = rgmsh.Remesh(...)
    # Approximation of contact length
    fine_x = closestPoint()
    # Definition of the field of element sizes
    mesh.field.setAsBackgroundMesh(bacFieldNum)
    # Remesh the deformable geometry
    mesh.generate(2)
    gmsh.write("geomwp.unv")
    gmsh.finalize()
    # Make the non-deformable geometry
    gmsh.initialize()
    base_face = geo.addPlaneSurface([base_loop])
    # Definition of the field of element sizes
    bsFields = rgmsh.BoxFieldsIncreasingSize(...fine_x...)
    # Remesh the non-deformable geometry
    mesh.generate(2)
    gmsh.write("geombs.unv")
    gmsh.finalize()
\end{minted}
% }}}
% }}}
% }}}

% BioInspiredGrowth.dumm {{{
\section{Code\_Aster script: Bio--inspired growth}
% Execution {{{
\subsection{Execution}
\begin{minted}{bash}
    .../salome shell -- as_run setup.export
\end{minted}
% }}}
% Required packages {{{
\subsection{Required packages}
\begin{minted}{python}
    # Python libraries
    import os
    import math
    import sys
    import json
    import numpy as np
    import time
    # Code_Aster libraries
    from Utilitai.partition import *
    # In-house modules
    sys.path.append('../PythonUtilities')
    import MorphoDesignFunctions as mdf
    from FemMesh2D import FemMesh2D as fm2
    from FemVtk import FemVtk as fvt
\end{minted}
% }}}
% General function {{{
\subsection{General function}
\begin{minted}{python}
    # Read parameters
    params = json.load("PARAMS.json")
    # Definition of Code_Aster static concepts: material, mesh and model
    mesh_wp = LIRE_MAILLAGE(FORMAT = 'IDEAS', ...)
    mesh_bs = LIRE_MAILLAGE(FORMAT = 'IDEAS', ...)
    mesh    = CREA_MAILLAGE(DECOUPE_LAC = _F(...), ...)
    mode    = AFFE_MODELE(AFFE = _F(MODELISATION = ('D_PLAN'), ...)
    mater   = DEFI_MATERIAU(ELAS = (...))
    matf    = AFFE_MATERIAU(AFFE = _F(MATER= mater, ...), MODELE = mode)
    # Set time--dependent control variables
    ite = ...
    bestContPre = ...
    # Define initial soft field (where the material can grow)
    toGrw0 = CREA_CHAMP(...)
    # Adjust the location of the geometries (necessary to ensure contact
    # convergence)
    if not interpenetration:
        reloc = CREA_CHAMP(...)
        mesh  = MODI_MAILLAGE(...)
    # Set up mesh and define loads
    mesh    = DEFI_GROUP(...)
    ...     = AFFE_CHAR_MECA(...)
    contact = DEFI_CONTACT(...)
    # Definition of formulas and function fields
    ... = FORMULE(...)
    ... = CREA_CHAMP(...)
    # Bio--inspired growth iterations
    for k1 in range(number of iterations):
        # Compute the stress field that induces growth (with contact
        # conditions)
        resu = STAT_NON_LINE(...)
        # Compute the hydrostatic and the shear stress
        resu = CALC_CHAMP(CONTRAINTE = ('SIGM_NOEU', 'SIGM_ELGA',
                                        'SIEF_NOEU'),
                          CRITERES = ('SIEQ_ELGA'),...)
        resu = CALC_CHAMP(CHAM_UTIL = _F(FORMULE = (Shyd_f, Svon_f),...)
        # Compute the contact pressure and its quality
        cont_p = CALC_PRESSION(...)
        Rp = mdf.Rp_OrderedData(...)
        # Compute the quality of the shear stress
        Rv = 1.0 - (maxTau - tauRef)/(maxTau + tauRef)
        # Compute the growth function
        Sgrw_f  = FORMULE(func = mdf.SgrowthDesign, ...)
        resu    = CALC_CHAMP(CHAM_UTIL = (_F(FORMULE = (Sgrw_f), ...)))
        grw_fi = CREA_CHAMP(OPERATION = 'EXTR', ...)
        toGrw  = CREA_CHAMP(OPERATION = 'EVAL', ...)
        # Get field norm
        grwStrVtk = fvt(...)
        grwNorm = grwStrVtk.LpNormScalar(0)
        # Scale field with the norm
        toGrwSca = CREA_CHAMP(OPERATION = 'COMB', ...)
        # Compute growth stress
        growth_t = CREA_CHAMP(OPERATION = 'ASSE', ...)
        # Compute the displacements driven by the growth stress
        resu = MECA_STATIQUE(...)
        # Apply displacements
        disp_ap1 = CREA_CHAMP(OPERATION = 'EXTR', ...)
        disp_ap2 = CREA_CHAMP(OPERATION = 'EVAL', ...)
        disp_ap3 = CREA_CHAMP(OPERATION = 'ASSE', ...)
        mesh = MODI_MAILLAGE(reuse = mesh, DEFORME = (disp_ap3))
\end{minted}
% }}}
% }}}

\end{document}
